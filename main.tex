\documentclass[
	% -- opções da classe memoir --
	article,			% indica que é um artigo acadêmico
	11pt,				% tamanho da fonte
	oneside,			% para impressão apenas no recto. Oposto a twoside
	a4paper,			% tamanho do papel. 
	% -- opções da classe abntex2 --
	%chapter=TITLE,		% títulos de capítulos convertidos em letras maiúsculas
	%section=TITLE,		% títulos de seções convertidos em letras maiúsculas
	%subsection=TITLE,	% títulos de subseções convertidos em letras maiúsculas
	%subsubsection=TITLE % títulos de subsubseções convertidos em letras maiúsculas
	% -- opções do pacote babel --
	english,			% idioma adicional para hifenização
	brazil,				% o último idioma é o principal do documento
	sumario=tradicional
	]{abntex2}


% ---
% PACOTES
% ---

% ---
% Pacotes fundamentais 
% ---
\usepackage[T1]{fontenc}
\usepackage[sfdefault]{AlegreyaSans} %% Option 'black' gives heavier bold face
%% The 'sfdefault' option to make the base font sans serif
\renewcommand*\oldstylenums[1]{{\AlegreyaSansOsF #1}}
\renewcommand{\ABNTEXchapterfont}{\fontseries{b}\selectfont}
\usepackage[utf8]{inputenc}		% Codificacao do documento (conversão automática dos acentos)
\usepackage{indentfirst}		% Indenta o primeiro parágrafo de cada seção.
\usepackage{nomencl} 			% Lista de simbolos
\usepackage{color}				% Controle das cores
\usepackage{graphicx}			% Inclusão de gráficos
\usepackage{microtype} 			% para melhorias de justificação
\usepackage[alf]{abntex2cite}	% Citações padrão ABNT
% ---
		
% ---
% Pacotes adicionais, usados apenas no âmbito do Modelo Canônico do abnteX2
% ---
\usepackage{lipsum}				% para geração de dummy text
% ---
		
% ---
% Pacotes de citações
% ---
\usepackage[brazilian,hyperpageref]{backref}	 % Paginas com as citações na bibl

% ---

% --- Informações de dados para CAPA e FOLHA DE ROSTO ---
\titulo{A leitura em papel e em dispositivos digitais e sua influência na compreensão de histórias em quadrinhos}

\autor{
Jônathas Souza\thanks{Departamento de Design, Centro de Artes e Comunicação, Universidade Federal de Pernambuco. Endereço para contato: \url{jonathas.barrossouza@ufpe.br}},  
Ana Beatriz Coutinho\thanks{Departamento de Design, Centro de Artes e Comunicação, Universidade Federal de Pernambuco} e Matheus Alencar\thanks{Departamento de Design, Centro de Artes e Comunicação, Universidade Federal de Pernambuco}}

\local{Brasil}
\data{2018}
% ---

% ---
% Configurações de aparência do PDF final

% alterando o aspecto da cor azul
\definecolor{blue}{RGB}{41,5,195}

% informações do PDF
\makeatletter
\hypersetup{
     	%pagebackref=true,
		pdftitle={\@title}, 
		pdfauthor={\@author},
    	pdfsubject={A leitura em papel e em dispositivos digitais e sua influência na compreensão de histórias em quadrinhos},
	    pdfcreator={LaTeX with abnTeX2},
		pdfkeywords={abnt}{latex}{abntex}{abntex2}{atigo científico}, 
		colorlinks=true,       		% false: boxed links; true: colored links
    	linkcolor=blue,          	% color of internal links
    	citecolor=black,        		% color of links to bibliography
    	filecolor=magenta,      		% color of file links
		urlcolor=blue,
		bookmarksdepth=4
}
\makeatother
% --- 

% ---
% compila o indice
% ---
\makeindex
% ---

% ---
% Altera as margens padrões
% ---
\setlrmarginsandblock{3cm}{3cm}{*}
\setulmarginsandblock{3cm}{3cm}{*}
\checkandfixthelayout
% ---

% --- 
% Espaçamentos entre linhas e parágrafos 
% --- 

% O tamanho do parágrafo é dado por:
\setlength{\parindent}{1.3cm}

% Controle do espaçamento entre um parágrafo e outro:
\setlength{\parskip}{0.2cm}  % tente também \onelineskip

% Espaçamento simples
\OnehalfSpace


% ----
% Início do documento
% ----
\begin{document}

% Seleciona o idioma do documento (conforme pacotes do babel)
%\selectlanguage{english}
\selectlanguage{brazil}

% Retira espaço extra obsoleto entre as frases.
\frenchspacing 

% ----------------------------------------------------------
% ELEMENTOS PRÉ-TEXTUAIS
% ----------------------------------------------------------

%---
%
% Se desejar escrever o artigo em duas colunas, descomente a linha abaixo
% e a linha com o texto ``FIM DE ARTIGO EM DUAS COLUNAS''.
% \twocolumn[    		% INICIO DE ARTIGO EM DUAS COLUNAS
%
%---

% página de titulo principal (obrigatório)
\maketitle


% titulo em outro idioma (opcional)



% resumo em português
\begin{resumoumacoluna}
 \lipsum[150]
 
 \vspace{\onelineskip}
 
 \noindent
 \textbf{Palavras-chave}: latex. abntex. editoração de texto.
\end{resumoumacoluna}


% resumo em inglês
\renewcommand{\resumoname}{Abstract}
\begin{resumoumacoluna}
 \begin{otherlanguage*}{english}
   \lipsum[150]

   \vspace{\onelineskip}
 
   \noindent
   \textbf{Keywords}: latex. abntex.
 \end{otherlanguage*}  
\end{resumoumacoluna}

% ]  				% FIM DE ARTIGO EM DUAS COLUNAS
% ---

\begin{center}\smaller
\textbf{Data de submissão e aprovação}: elemento obrigatório. Indicar dia, mês e ano
\end{center}

% ----------------------------------------------------------
% ELEMENTOS TEXTUAIS
% ----------------------------------------------------------
\textual

% ----------------------------------------------------------
% Introdução
% ----------------------------------------------------------
\section{Introdução}

\lipsum[55-57]
% ----------------------------------------------------------
% Seção de explicações
% ----------------------------------------------------------
\section{Revisão de Literatura}

Ler é um ato extremamente importante do ponto de vista sociocultural. \textcolor{red}{\textbf{FALAR AQUI DE ALFABETISMO, ANALFABETISMO E ANALFABETISMO FUNCIONAL}}. O ato de ler também tem consequências que ultrapassam sua tarefa imediata de absorver sentido de uma passagem, tendo implicações profundas no desenvolvimento de um grande número de capacidades cognitivas \cite{Cunningham2001}.
Entretanto, a concepção de leitura tem passado por grandes modificações, a partir do aparecimento das mídias digitais e do maior consumo de texto a partir de dispositivos como e-readers (ex. Kindle) e tablets (ex. iPad), que possuem características completamente diferentes das do texto impresso \cite{Mangen2016}. Esse número crescente de novas mídias forçou, inclusive, novas interpretações de termos como “texto”, “leitura” e “alfabetização”, levando ao surgimento do conceito de “alfabetização digital”, que representa “práticas socialmente situadas e suportadas por habilidades, estratégias e instâncias que permitem a representação e o entendimento de ideias utilizando uma variedade de modalidades possibilitadas por ferramentas digitais” \cite{OBrien2008}.

A leitura também é vista como um processo de alta concentração. Levy relembra as principais representações do momento da leitura: um no início da vida, quando a criança tenta dar seus primeiros passos na leitura de um texto, empregando total concentração e esforço, e outro durante a vida, nos momentos de estudo e leitura concentrada e silenciosa. A própria imagem da biblioteca silenciosa denota esta percepção \cite{Levy1997}. Entretanto, a multiplicidade de mídias e dispositivos levou a sociedade a adquirir um estado de atenção dividida, onde a leitura tem se tornado menos imersiva e demandado maior concentração do que jogar, assistir ou ouvir, principalmente por essas atividades serem acessadas nos mesmos dispositivos \cite{Mangen2016}.

A mídia digital contribui para uma mudança radical na leitura, além de introduzir uma série de vantagens que estão tradicionalmente ausentes na versão impressa de um texto, tais como interatividade, não-linearidade, acesso imediato à informação secundária, e a convergência de texto e imagens, audio e video \cite{Liu2005}

 	Mangen faz questão de 


\section{Metodologia}

\lipsum[55-55]

\section{Resultados}

\subsection{Experimento 1}

\lipsum[55]

\subsection{Experimento 2}

\lipsum[55]

\section{Discussão}

\lipsum[55]
% ---
% Finaliza a parte no bookmark do PDF, para que se inicie o bookmark na raiz
% ---
\bookmarksetup{startatroot}% 
% ---

% ---
% Conclusão
% ---
\section{Considerações finais}

\lipsum[1]

\begin{citacao}
\lipsum[2]
\end{citacao}

\lipsum[3]

% ----------------------------------------------------------
% ELEMENTOS PÓS-TEXTUAIS
% ----------------------------------------------------------
\postextual

% ----------------------------------------------------------
% Referências bibliográficas
% ----------------------------------------------------------

\bibliography{references}


% ----------------------------------------------------------
% Agradecimentos
% ----------------------------------------------------------

\section*{Agradecimentos}
Texto sucinto aprovado pelo periódico em que será publicado. Último 
elemento pós-textual.

\end{document}