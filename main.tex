\documentclass{article}
\usepackage[utf8]{inputenc}
\usepackage[T1]{fontenc}
\title{Scientific Reports Title to see here}

\author[1,*]{Alice Author}
\author[2]{Bob Author}
\author[1,2,+]{Christine Author}
\author[2,+]{Derek Author}
\affil[1]{Affiliation, department, city, postcode, country}
\affil[2]{Affiliation, department, city, postcode, country}

\affil[*]{corresponding.author@email.example}

\affil[+]{these authors contributed equally to this work}

%\keywords{Keyword1, Keyword2, Keyword3}

\begin{abstract}
Example Abstract. Abstract must not include subheadings or citations. Example Abstract. Abstract must not include subheadings or citations. Example Abstract. Abstract must not include subheadings or citations. Example Abstract. Abstract must not include subheadings or citations. Example Abstract. Abstract must not include subheadings or citations. Example Abstract. Abstract must not include subheadings or citations. Example Abstract. Abstract must not include subheadings or citations. Example Abstract. Abstract must not include subheadings or citations.
\end{abstract}


\begin{document}

\maketitle

\section{Introduction}
There is a theory which states that if ever anyone discovers exactly what the Universe is for and why it is here, it will instantly disappear and be replaced by something even more bizarre and inexplicable.
There is another theory which states that this has already happened.

\begin{figure}[h!]
\centering
\includegraphics[scale=1.8]{universe}
\caption{The Universe}
\label{fig:universe}
\end{figure}

\section{Conclusion}
``I always thought something was fundamentally wrong with the universe'' \citep{adams1995hitchhiker}

\bibliographystyle{plain}
\bibliography{references}
\end{document}
